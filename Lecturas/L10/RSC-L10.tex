\documentclass[a4paper, 12pt]{report}
\usepackage[spanish]{babel}
\usepackage[utf8]{inputenc}
\usepackage{textcomp}
\usepackage{booktabs}
\usepackage{amssymb}
\usepackage{bussproofs}
\usepackage{proof}

\usepackage{geometry}
\geometry{
 top=4cm,
}

\usepackage{fancyhdr}
\usepackage{graphicx}
\usepackage{amsmath}

\pagestyle{fancy}
\lhead{jalrod@ciencias.unam.mx}
\rhead{jeanpaul@ciencias.unam.mx}

\begin{document}

\begin{flushright}
    Almeida Rodríguez Jerónimo\\
    Ruiz Melo Jean Paul
\end{flushright}

\begin{center}
    {\LARGE Lectura 10}\\
    {\LARGE Collective dynamics of ‘small-world’ networks}
\end{center}
Podemos ver el relacion entre redes regulares y redes aleartorias usando redes de 
tipo mundo pequenio. Empezamos suponiendo que hay n vertices, un numero de aristas 
k por vertice, con una probabilidad p para mover una connecion, tal que esta entre
0 y 1. Tomamos a p=0 como regularidad y p=1 como disorden. \\

Podemos ahora cuantificar las caracteristicas de un red tomando la distancia 
caracteristica entre nodos como L(p) y el coeficiente de agrupamiento como C(p).
Sabemos que L(p) es un propiedad global, mientras que C(p) es una propiedad local.
Los redes que son mejor trabajar para ver esto son los que tienen los siguientes 
propiedades:
\begin{center}
    1 $<$ ln(n) $<$ k $<$ n
\end{center}
Este propiedad asegura que el red tiene conexiones escasas sin que esta disconexa.
Viendo los redes ahora, se puede notar que un p alto resulta en un C(p) bajo, y un
p bajo resulta en un C(p) alto. Pero tiene el relacion opuesto a respeto de L(p), 
ya que da un L(p) alto y un L(p) bajo respectivamente. Esto es debido a que a tomar
una connecion cercana y moverlo con uno mas lejos, resulta que hace un atajo entre 
dos vertices que baja la distancia entre vertices en general.  \\

Esto tiene un resultado que no se nota mucho en redes adrupadas, ya que tiene un 
efecto lineal, mientras que la distancia se baja exponencialmente. Entonces el 
transicion a un mundo pequenio resulta ser un poco sutible, donde se tiene que 
ver mas por las caracteristicas de conexiones que tiene el red.\\

Por ejemplo, cuando se tiene un p muy bajo, el infectividad de un enfermedad se 
hace mas lento, ya que aunque hay un clustering, no se puede propgar al otro 
lado del red muy rapido, pero con p muy alto, puede hacer estos brincos de manera
mas rapido.Encontes aunque estos resultados pueden difereir de otros modelos, que
notan que la estructura del red tiene un impacto sobre el transmision, ilustra 
que la dinamica es un funcion del estructura. \\

Aunque los redes de mundos pequenios no han recibido mucho atencion, pueden ser
importantes con un enfoque mas en los redes biologicos, sociales y hechos por 
humanos con importancia dinamico.

\end{document}
