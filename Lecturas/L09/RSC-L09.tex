\documentclass[a4paper, 12pt]{report}
\usepackage[spanish]{babel}
\usepackage[utf8]{inputenc}
\usepackage{textcomp}
\usepackage{booktabs}
\usepackage{amssymb}
\usepackage{bussproofs}
\usepackage{proof}

\usepackage{geometry}
\geometry{
 top=4cm,
}

\usepackage{fancyhdr}
\usepackage{graphicx}
\usepackage{amsmath}

\pagestyle{fancy}
\lhead{jalrod@ciencias.unam.mx}
\rhead{jeanpaul@ciencias.unam.mx}

\begin{document}

\begin{flushright}
    Almeida Rodríguez Jerónimo\\
    Ruiz Melo Jean Paul
\end{flushright}

\begin{center}
    {\LARGE Lectura 9}\\
    {\LARGE The Beasts That Keep the Beat}
\end{center}

Se ha notado que varios animales tienen una capacidad de entrenar
sus movimientos a un ritmos externo. Originalmente se dio cuente con
un cacatúa que se llamaba Snowball, cuando se subió un video del aquel
bailando. \\

Después de este evento, se empezando a hacer más experimentos al respecto
con varios animales diferentes. Originalmente se puso el hipótesis de que
solo animales que tenia una capacidad de aprendizaje vocal podían aprender a
'bailar'. Particularmente que animales como perros o caballos no podían hacerlo,
mientras animales como elefantes, pájaros o humanos si lo podrían hacer.\\

Esto se empezó a ver con un león marino que se llamaba Ronan. Le mostraron
unos pulsos tipo metrónomo, los cuales ella movió su cabeza al ritmo dé.
Después, le dieron unos pulsos que no le habían enseñado, y resulta que 
tambien podría mover su cabeza a tiempo. Esto rompió el hipótesis que solo los
animales que tenían un aprendizaje vocal podrían bailar.\\

A partir de esto, se empezó a ver que otros animales que no se pensaban podrían
hacerlo, podían aprender a mover su cuerpo a un ritmo. Aunque hay científicos que
quieren que se hacen con perros y caballos que son animales que notablemente no eran
capacitados de aprendizaje vocal.\\

De esto se empezó a preguntar que entonces como el cerebro puede seguir un ritmo.
Sabemos que los cerebros que todos los animales son machinas biológicos de alto
ritmo, entonces podemos usar redes de electrodos para poder ver fluctuaciones
en el cerebro.\\

Con esto, se podía entonces ver que todo los animales si podían seguir un ritmo,
solo que algunos animales como los pájaros y elefantes, que son muy sociables
ya que replican sonidos que han escuchado antes, se les hace más fácil para hacerlo.
Otros animales tienen la capacidad, solo que no les interesa hacerlo. Ronan por ejemplo,
solo lo empezó a hacer cuando fue entrenado. \\

Ale mejor no somos tan especial que podemos hacer música, pero que es más fácil
porque lo tenemos en todo los pasos de nuestro vida, desde que nacimos hasta
que morimos.

\end{document}
