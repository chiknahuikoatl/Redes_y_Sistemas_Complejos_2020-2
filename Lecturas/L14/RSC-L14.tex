\documentclass[a4paper, 12pt]{report}
\usepackage[spanish]{babel}
\usepackage[utf8]{inputenc}
\usepackage{textcomp}
\usepackage{booktabs}
\usepackage{amssymb}
\usepackage{bussproofs}
\usepackage{proof}

\usepackage{geometry}
\geometry{
 top=4cm,
}

\usepackage{fancyhdr}
\usepackage{graphicx}
\usepackage{amsmath}

\pagestyle{fancy}
\lhead{jalrod@ciencias.unam.mx}
\rhead{jeanpaul@ciencias.unam.mx}

\begin{document}

\begin{flushright}
    Almeida Rodríguez Jerónimo\\
    Ruiz Melo Jean Paul
\end{flushright}

\begin{center}
    {\LARGE Lectura 14}\\
    {\LARGE A Network Theory Analysis of Football Strategies.}
\end{center}

A diferencia de otros deportes, el fútbol no se había prestado mucho a mostrar
análisis de los juegos. Fue hasta hace relativamente poco que se comenzaron a
sacar datos estadísticos sobre los juegos. Una manera de medir los juegos es
haciendo una gráfica de pases.\\

El artículo describe la red dirigida de un equipo basándose en los pases. Cada
jugador es un nodo y las flechas existen si un jugador da un pase a otro
jugador. Se le puede dar peso a las flechas basándose en el número de pases que
hay de un jugador a otro y en la certeza de los pases. La distancia entre dos
nodos está dada por $l_{i,j}=\frac{1}{A_{i,j}}$ dónde $A$ es la matriz de
adyacencia con peso e $i\neq j$. Se busca entonces la menor distancia (no
necesariamente física) entre cualesquiera dos nodos. Esto nos da una idea de la
conectividad en las flechas lo cual es importante conocer para poder medir la
robustez de la red.\\

Algunas medidas que nos ayudan a medir lo antes mencionado son:
\begin{itemize}
    \item{\textbf{Centralidad de Cercanía}: Esto nos ayuda a medir qué tan fácil
        es llegar a un jugador dentro de la gráfica.
    }
    \item{\textbf{``Betweenness''}: Esta medida nos muestra a que grado un nodo
        conecta cualesquiera otros dos. Lo que mide es cómo se mueve el balón
        entre jugadores y qué tanto afectaría que ese jugador no estuviera.
    }
    \item{\textbf{Popularidad}: Qué tantos pases recibe un jugador de otros
        jugadores.
    }
\end{itemize}

Otro aspecto importante a analizar es el clustering, que muestra los nodos
centrales para llegar de un nodo a otro. Por otra parte, estudiar los cliques
ayuda a ver que tan ``unido'' es el equipo en términos de grupos de jugadores
que se pasan entre sí el balón. Un equipo bien conectado tiene un clique maximal
grande que muestra que hay pases entre sí de casi cualesquiera dos jugadores.\\

Los datos del mundial analizados muestran que los equipos con un ``betweenness''
distribuido normalmente y un clique grande resultaron ser más exitosos. Según el
artículo, los equipos que menos tienden a confiar en un número pequeño de
jugadores tiene mayor probabilidad de ganar.

\end{document}
