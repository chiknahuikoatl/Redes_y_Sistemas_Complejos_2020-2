\documentclass[a4paper, 12pt]{report}
\usepackage[spanish]{babel}
\usepackage[utf8]{inputenc}
\usepackage{textcomp}
\usepackage{booktabs}
\usepackage{amssymb}
\usepackage{bussproofs}
\usepackage{proof}

\usepackage{geometry}
\geometry{
 top=4cm,
}

\usepackage{fancyhdr}
\usepackage{graphicx}
\usepackage{amsmath}

\pagestyle{fancy}
\lhead{jalrod@ciencias.unam.mx}
\rhead{jeanpaul@ciencias.unam.mx}

\begin{document}

\begin{flushright}
    Almeida Rodríguez Jerónimo\\
    Ruiz Melo Jean Paul
\end{flushright}

\begin{center}
    {\LARGE Lectura 16}\\
    {\LARGE Forests Emerge as a Major Overlooked Climate Factor.}
\end{center}

Por décadas los científicos han desarrollado sus modelos de predicción climática
basándose en viento, lluvia y otros fenómenos físicos, pero con el avance de la
computación y la capacidad de simular cómo las plantas mueven agua, dióxido
de carbono y otros componentes entre el aire y la tierra, se han comenzado a
estudiar los efectos globales que puede tener un bosque.\\

Las plantas absorben dióxido de carbono y energía solar de la atmósfera y agua
del suelo y forman azúcares, oxigeno y agua, la cual es reintroducida a la
atmosfera. La unión de una gran cantidad de plantas (en un bosque) es lo que
causa los efectos a gran escala. Por ejemplo, mucha del agua que recibe San
Pablo, Brasil viene de los ``ríos voladores'' producidos por la mata
amazónica.\\

No obstante, se ha observado que estos efectos no son locales, i.e. que
solamente afectan a una región circundante, sino que pueden tener efectos al
otro lado del planeta. En unas simulaciones que hizo la dra. Abigail Swann sobre
el efecto que tendría reforestar la tundra ártica observó que en esa zona la
temperatura incrementaría, lo cual causaría más deshielo y a su vez daría lugar
a una zona boscosa más grande, que a su vez generaría un ciclo de
retroalimentación sobre dicha zona. Por ejemplo, la dra. Swann notó que al mover
la célula de Hadley por medio de reforestar Norte América, Europa y Asia las
sequías se reducían en el sur del Amazonas y la lluvia incrementaba en el
Sahara. El otro lado de la moneda es que se ha observado que una reducción del
$20\%$ de la mata Amazónica ha disparado las sequías en California. Por otro
lado, la dra. Swann descubrió que quitar los bósques del Atlántico medio
permite que otros bosques sean más templados y húmedos. Sin embargo, el efecto
global que tendrían las reforestaciones masivas en China y el Sahel de África
aún se desconocen.\\

La principal fuente de incertidumbre en modelos climáticos han sido las nubes
pues reflejan la radiación solar pero también calientan la superficie de la
tierra. Ahora, gracias a los estudios de la dra. Swann, los bosques también
pueden ser tomados en cuenta cómo fuentes de incertidumbre. Más que ver nubes y
bosques cómo fenómenos distintos, se deberían ver cómo un conjunto, pues los
bosques generan en gran parte a las nubes.

\end{document}
