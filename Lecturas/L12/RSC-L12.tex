\documentclass[a4paper, 11pt]{report}
\usepackage[spanish]{babel}
\usepackage[utf8]{inputenc}
\usepackage{textcomp}
\usepackage{booktabs}
\usepackage{amssymb}
\usepackage{bussproofs}
\usepackage{proof}

\usepackage{geometry}
\geometry{
 top=4cm,
}

\usepackage{fancyhdr}
\usepackage{graphicx}
\usepackage{amsmath}

\pagestyle{fancy}
\lhead{jalrod@ciencias.unam.mx}
\rhead{jeanpaul@ciencias.unam.mx}

\begin{document}

\begin{flushright}
    Almeida Rodríguez Jerónimo\\
    Ruiz Melo Jean Paul
\end{flushright}

\begin{center}
    {\LARGE Lectura 12}\\
    {\LARGE Global Network of Cargo Ships}
\end{center}

El articulo habla de redes de puertos que están conectados si un barco
pasa por ellos. El estudio muestra una
topologia de la red mundo pequeño con una distribucion de cola larga y busca
los puertos centrales al red y de esos puertos interconectados para mostrar la
importancia de regiones geopoliticas y bloques de comercio. Hay tres ``capas''
para analizar esta red las cuales representan diferentes tipos de barcos, cada
uno con patrones de rutas y efectos ecológicos distintos\\

Primero, consideramos las trayectorias que toma un barco entre puerto y puerto
cómo la red. Redes mas grandes se pueden tomar como las uniones de varias
trayectorias. Luego, para este estudio, se toman 4 redes: las subredes de los
porta-contenedores, graneleros secos y petroleros y una todas las
trayectorias. Se asigna peso $w$ entre los puertos i y j como la capacidad de
todos los barcos que han viajado por esa conexion. \\

En la red de porta-contenedores, se nota que cualesquiera dos puertos de
una componente gigante tienen una ruta corta para llegar a cualquier
otro puerto. El tamaño de ruta promedio es de 2.5 pasos a pesar de que las rutas
circulares son la norma. Cómo el coeficiente de clustering $C=0.49$ la red se
considera de mundo pequeño.\\

La distribucion de grados muestra que la mayoría de los puertos
tienen pocas conexiones, pero hay algunos que tienen hasta cientos de conexiones.
La distribución de los pesos de aristas sigue un regla de P(w) $<$ w$^u$. La
fuerza de un puerto (P(s) $<$ s$^n$ con n = 1.02 $\pm$ 0.17, el peso
entrando y saliendo de un puerto) suele crecer mas rapido que su grado, debido a
una ley de escala. Otra característica importante es la cercanía de centralidad
de un puerto (cuántos caminos pasan por él).\\

Para ver las caracteristicas de cada puerto en las redes, podemos ver
sus comunidades (puertos que comparten caminos). Se puede perder algunas
comunidades debido a lo facil que se
detectan debido a optimizaciones de modularidad, pero no afecta la
informacion registrada de comundades. Una cosa que se comparte
entre las redes es la distribucion de 3-motivos, que se da principalmente por
el robustez y conectivadad del red. \\

Podemos tomar un indice regularidad p que difiere un arista de una red
arbitraria para saber que tipo de barco viaja mas. Si un barco viaja S veces a N
puertos distintos siguiendo L aristas, podemos comparar el promedio de viajes por
arista f$_{real}$ = S / L con las trayectorias aleatorias f$_{ran}$. Podemos
calcular entonces la diferencia entre ellas con
Z = $\frac{(f_{real} - f_{ran})}{o}$, o = desviacion estándar. Si este Z = 0,
entonces se toman trayectorias arbitrarias. \\



\end{document}
