\documentclass[a4paper, 12pt]{report}
\usepackage[spanish]{babel}
\usepackage[utf8]{inputenc}
\usepackage{textcomp}
\usepackage{booktabs}
\usepackage{amssymb}
\usepackage{bussproofs}
\usepackage{proof}

\usepackage{geometry}
\geometry{
 top=4cm,
}

\usepackage{fancyhdr}
\usepackage{graphicx}
\usepackage{amsmath}

\pagestyle{fancy}
\lhead{jalrod@ciencias.unam.mx}
\rhead{jeanpaul@ciencias.unam.mx}

\begin{document}

\begin{flushright}
    Almeida Rodríguez Jerónimo\\
    Ruiz Melo Jean Paul
\end{flushright}

\begin{center}
    {\LARGE Lectura 8}\\
    {\LARGE Sincronía Sintomática}
\end{center}

Una de las preguntas que más han entretenido a los científicos de la medicina es
la cuestión de el ritmo de gestación de una enfermedad. ¿Cómo es que una persona
se enferma y en cuestión de días explota la enfermedad dentro de su comunidad?
Los modelos han mostrado que el tiempo de gestación se puede ver cómo una
distribución de cola larga.\\

Un equipo de investigación conformado por un matemático y un médico científico
investigando la proliferación de virus desarrolló un modelo basado en geometría
y probabilida para encontrar la dsitribución de dicha enfermedad. Aunque estos
modelos no se sostienen bajo el escrutinio matemático, arrojan cómo resultado
distribuciones parecidas a las que se han encontrado en la vida real.\\

El acercamiento que tiene este modelo es ver a cada célula sana dentro de una
red con otras células sanas cómo vecinas. Cada vez que una célula muere, las
células a su alrededor (sanas y enfermas) intentan llenar el lugar que dejó.
Entonces, una célula enferma puede ir ``ganando terreno'' adyacente mientras sus
vecinos mueren. la distribución de cola larga se hace evidente en el punto en el
que las células se han proliferado al punto de que quedan pocas células sanas
para suplatar además de las limitaciones físicas del anfitrión; esto alenta la
proliferación de la enfermedad.\\

A pesar de que estos modelos no son completamente acertados biológicamente, dan
una aproximación lo suficientemente cercana para poder encontrar el periodo
ideal para tratar dichas enfermedades antes de que estas se proliferen. Por
ejemplo, se sabe que si un patógeno crece exponencialmente, entonces
estadísticamente la exposición de una población es representada por una
distribución normal. De esta manera, se puede dar una mejor atención a la
población.


\end{document}
