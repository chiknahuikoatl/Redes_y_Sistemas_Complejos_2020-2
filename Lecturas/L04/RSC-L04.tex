\documentclass[a4paper, 12pt]{report}
\usepackage[spanish]{babel}
\usepackage[utf8]{inputenc}
\usepackage{textcomp}
\usepackage{booktabs}
\usepackage{amssymb}
\usepackage{bussproofs}
\usepackage{proof}

\usepackage{fancyhdr}
\usepackage{graphicx}
\usepackage{amsmath}

\pagestyle{fancy}
\lfoot{jalrod@ciencias.unam.mx}
\rfoot{jeanpaul@ciencias.unam.mx}

\begin{document}

\begin{flushright}
    Almeida Rodríguez Jerónimo\\
    Ruiz Melo Jean Paul
\end{flushright}

\begin{center}
    {\LARGE Lectura 4}\\
    {\LARGE La Inteligencia de los Áboles}
\end{center}

Recientemente se ha redescubierto que los árboles tienen más conecciones de las
que podemos percibir a simple vista. En un bosque los árboles interactúan entre
sí por medio de mensajes químicos que se transimiten entre sí por medio de sus
raíces. Estos mensajes afectan a toda la ``red'' de árboles que están conectados
entre ellos al punto en el que si un árbol está en alguna clase de peligro
pronto todo el bosque lo sabe y cada árbol desprende un olor característico, es
decir, hay un cambio físico a partir de lo que se puede interpretar cómo una
``precepción personal'' de un árbol.\\

La comunicación entre los árboles puede llegar a ser tan profunda que según la
dra. Simard un árbol puede reconocer a su retoño y dependiendo si se encuentra
en una situación precaria (alguno de los dos árboles esté enfermo, le falten
nutrientes, etc.) puede favorecerle otorgándole más nutrientes. Inclusosi hay
otro retoño que la planta madre no reconoce cómo suyo siempre dará prioridad a
su propio retoño.\\

La dra. Simard hace el símil de esta red de árboles con una red neuronal. Ella
hace énfasis en que hay que tener cuidado con la definición pues la inteligencia
es subjetiva a la definición; pero entonces una red de árboles se puede enterar
si están en peligro o beneficiarse entre ella por medio de la comunicación (otro
término definido muy libremente) que se puede dar entre ellos.\\

Aunque es evidente que existe esta conección entre un sistema completo de
árboles, la evidencia no es siempre tan clara en árboles individuales. Esto se puede ver en el grosor de las raíces de un árbol.  El hecho de que estos
sistemas sean de libre escala (hay nodos muy grandes y también muy pequeños) ha
permitido una evolución hacia la resistencia y eficiencia.

\end{document}
