\documentclass[a4paper, 12pt]{report}
\usepackage[spanish]{babel}
\usepackage[utf8]{inputenc}
\usepackage{textcomp}
\usepackage{booktabs}
\usepackage{amssymb}
\usepackage{bussproofs}
\usepackage{proof}

\usepackage{fancyhdr}
\usepackage{graphicx}
\usepackage{amsmath}

\pagestyle{fancy}
\lfoot{jalrod@ciencias.unam.mx}
\rfoot{jeanpaul@ciencias.unam.mx}

\begin{document}

\begin{flushright}
    Almeida Rodríguez Jerónimo\\
    Ruiz Melo Jean Paul
\end{flushright}

\begin{center}
    {\LARGE Lectura 5}\\
    {\LARGE Complejidad}
\end{center}

La diferencia entre un sitema complejo y un sistema simple es el conjunto de
datos con el que se va a trabajar. En el caso del sistema complejo, esta
cantidad de datos puede ser abrumadora. La solución que se ha encontrado para
poder estudiar estos sistemas complejos es analizando esta información a
diferentes escalas, es decir, aplicando una granularidad distinta al objeto que
estamos estudiadiando.\\

El artículo usa cómo ejemplo el comportamiento físico del agua ejemplificando
que si bien, podemos estudiar el comportamiento de cada átomo en función de la
temperatura y presión, este análisises impráctico debido a que hay millones de
átomos en una muestra dada. Por otro lado, si ``escalamos hacia arriba'' el
análisis podemos observar que una masa de agua tiene cierto comportamiento
consistente dependiendo del cambio de temperatura y presión.\\

Este comportamiento puede modelarse matemáticamente de manera relativamente
sencilla hasta cierto punto, pues el comportamiento del agua, al cambiar de fase
diverge de dicha predicción. Esto nos habla de una de las limitaciones más
importantes de la modelación matemática; se asume que el comportamiento del
objeto de estudio es regular a lo largo del parámetro de estudio (en este caso
temperatura y presión pero también pude ser tiempo).\\

El análisis se puede hacer a escala macroscópica o microscópica. Una macro
escala da información general del sistema a partir de las divisiones más simples
que se pueden obtener del mismo. Por su parte, una micro escala particiona aún
más al sistema, lo cual trae consigo dos aspectos importantes:
\begin{enumerate}
    \item Se requiere más información para describir a los elementos de este
    sistema y por lo tanto es más pesado procesar esta información.
    \item Al describirse características básicas, se pueden definir distintos
    sistemas con los mismos elementos variando apenas un poco el comportamiento
    o el entorno.
\end{enumerate}
\end{document}
