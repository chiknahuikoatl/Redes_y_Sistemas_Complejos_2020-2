\documentclass[a4paper, 12pt]{report}
\usepackage[spanish]{babel}
\usepackage[utf8]{inputenc}
\usepackage{textcomp}
\usepackage{booktabs}
\usepackage{amssymb}
\usepackage{bussproofs}
\usepackage{proof}

\usepackage{fancyhdr}
\usepackage{graphicx}
\usepackage{amsmath}

\pagestyle{fancy}
\lfoot{jalrod@ciencias.unam.mx}
\rfoot{jeanpaul@ciencias.unam.mx}

\begin{document}

\begin{flushright}
    Almeida Rodríguez Jerónimo\\
    Ruiz Melo Jean Paul
\end{flushright}

\begin{center}
    {\LARGE Lectura 5}\\
    {\LARGE Complejidad}
\end{center}

La diferencia entre un sitema complejo y un sistema simple es el conjunto de
datos con el que se va a trabajar. En el caso del sistema complejo, esta
cantidad de datos puede ser abrumadora. La solución que se ha encontrado para
poder estudiar estos sistemas complejos es analizando esta información a
diferentes escalas, es decir, aplicando una granularidad distinta al objeto que
estamos estudiadiando.\\

El artículo usa cómo ejemplo el comportamiento físico del agua. En este ejemplo
podemos ver cómo las ideas de escala múltiple funcionan. Por ejemplo, no nos
enfocamos en el comportamiento individual de los átomos sino que por medio de la
presión, temperatura y volumen podemos describir lo que vemos y podemos
manipular. Granularizando esta escala podemos manejar los cambios en los niveles
más bajos sacando promedios de comportamiento. Esto nos ayuda a decidir qué
parámetros son importantes según y cómo se comportan en distintas escalas.\\

Este comportamiento puede modelarse matemáticamente de manera relativamente
sencilla hasta cierto punto, pues el comportamiento del agua, al cambiar de fase
diverge de dicha predicción, lo cual denota una de las limitaciones más
importantes de la modelación matemática que es asumir que el comportamiento del
objeto de estudio es regular.\\

Es importante notar que el modelo es un mapeo uno a uno con el sistema
para evitar pérdida de información. La información necesaria para representar un
sistema se conoce cómo el perfil de complejidad. La cantidad total de
información está definida cómo el log$_2$ del número de bits necesarios para
representar el número total de mensajes.\\

Al normalizar toda esta información se han encontrado instancias de sistemas con
el mismo comportamiento a gran escala. Este comportamiento se conoce cómo
universalidad de clase. Un ejemplo de esto son los patrones de Turing dónde el
comportamiento microscópico solamente sirve cómo parámetros para dar el
comportamiento general del patrón.

\end{document}
