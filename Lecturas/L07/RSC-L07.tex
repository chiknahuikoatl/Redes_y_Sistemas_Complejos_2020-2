\documentclass[a4paper, 12pt]{report}
\usepackage[spanish]{babel}
\usepackage[utf8]{inputenc}
\usepackage{textcomp}
\usepackage{booktabs}
\usepackage{amssymb}
\usepackage{bussproofs}
\usepackage{proof}

\usepackage{geometry}
\geometry{
 top=4cm,
}

\usepackage{fancyhdr}
\usepackage{graphicx}
\usepackage{amsmath}

\pagestyle{fancy}
\lhead{jalrod@ciencias.unam.mx}
\rhead{jeanpaul@ciencias.unam.mx}

\begin{document}

\begin{flushright}
    Almeida Rodríguez Jerónimo\\
    Ruiz Melo Jean Paul
\end{flushright}

\begin{center}
    {\LARGE Lectura 7}\\
    {\LARGE Altruismo en los Virus}
\end{center}

Muchos biólogos evolutivos intentan resolver la cuestión del comportmiento
altruista en algunas especies consideradas cómo sociales. Esto se puede ver en
especies consideradas hasta cierto punto complejas, cómo hormigas, abejas, etc.
Lo interesante es cuándo este comportamiento se presenta en organismos mucho más
simples cómo virus y bacterias.\\

Según un estudio realizado por Rafael Sanjuán en la universidad de Valencia,
España, la probabilidad de que un virus se reproduzca con éxito está en gran
parte dada por el grado de aislamiento que este virus posea con respecto a otros
organismos con una mayor habilidad reproductiva. De esto se concluye que un
virus aunque no tenga una ventaja reproductiva absoluta, aún puede tener éxito
evolutivo.\\

El caso de estudio es el virus VSV, un miembro mucho menos agresivo de la
familia de la rabia. Este virus, por un lado posee una habilidad reproductiva
reducida y por el otro lado tiene la habilidad de suprimir el interferón, una
proteína que ayuda a inhibir la reproducción de un virus. Este virus prospera en
un ambiente aislado gracias a la estructura del cuerpo de su anfitrión.\\

Para modelar las condiciones en las que la supresión inmunológica, los
científicos utilizan la regla de Hamilton: $r\times B > C$, dónde B es el
beneficio que obtiene el virus, r el nivel de relación con el anfitrión y C el
costo al anfitrión. De esto se encontró que si el virus se desarrolla aislado,
entonces prospera. De otro modo esta característica no puede evolucionar.\\

Así, pues, un virus puede beneficiarse de una alta tasa de reproducción en
etapas iniciales pero al final puede convenirle más reducir el ritmo de
reproducción.

\end{document}
