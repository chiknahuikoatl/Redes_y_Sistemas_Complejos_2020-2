\documentclass[a4paper, 12pt]{report}
\usepackage[spanish]{babel}
\usepackage[utf8]{inputenc}
\usepackage{textcomp}
\usepackage{booktabs}
\usepackage{amssymb}
\usepackage{bussproofs}
\usepackage{proof}

\usepackage{fancyhdr}
\usepackage{graphicx}
\usepackage{amsmath}

\pagestyle{fancy}
\lfoot{jalrod@ciencias.unam.mx}
\rfoot{jeanpaul@ciencias.unam.mx}

\begin{document}

\begin{flushright}
    Almeida Rodríguez Jerónimo\\
    Ruiz Melo Jean Paul
\end{flushright}

\begin{center}
    {\LARGE Lectura 6}\\
    {\LARGE Big Data}
\end{center}

En el siglo XXI muchos de los avances cinetíficos y descubrimientos que se han
hecho son gracias a una ``nueva'' rama de las matemáticas llamada teoría de
redes, la cual se ocupa de estudiar las relaciones entre los conjuntos de
información (nodos).\\

Según el artículo, una red es una entidad de partes interconectadas en dónde las
conecciones son más importantes que las partes en sí. Esto está definido por una
propiedad llamada complejidad que significa que todo está conectado entre sí, lo
cual puede causar variso problemas. A pesar de que este tipo
de estructuras y su estudio comenzaron a popularizarse con el desarrollo de
internet, las computadoras con grandes capacidades de cómputo, etre otras
tecnologías, estas se pueden encontrar casi en todos los campos de estudio;
desde interacción entre átomos hasta comercio mundial, pasando por interacción
molecular, la cadena alimenticia, redes áereas, etc.\\

Eventualmente, al estudiar estos sistemas uno se da cuenta de que se vuelven
extremadamente complejos; y ahí viene la belleza de la teoría de redes, pues toma
este sistema complejo y lo reduce a elementos y relaciones entre sí. Este poder
de abstracción es el que permitió a personas cómo Henry Beck simplificar el mapa
de red de trenes de Londres o a Kevin Bacon desarrollar el juego de ``Los Seis
Grados de Kevin Bacon''.\\

Aquí es dónde entra en juego el concepto de ``Big-data'' con el nacimiento de la
internet. El comportamiento de los usuarios en internet permitió a los
científicos descubrir reglas generales de comportamiento cómo los ``Hub's'' que
son nodos por dónde transita mucha información. Analizar estas cantidades
masivas de información ha otorgado a los científicos el poder de observar
sistemas complejos de arriba hacia abajo y poder encontrar reglas generales en
el comportamiento de estos sistemas que desde abajo pueden parecer caóticos.

\end{document}
