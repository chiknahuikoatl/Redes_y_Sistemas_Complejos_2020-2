\documentclass[a4paper, 12pt]{report}
\usepackage[spanish]{babel}
\usepackage[utf8]{inputenc}
\usepackage{textcomp}
\usepackage{booktabs}
\usepackage{amssymb}
\usepackage{bussproofs}
\usepackage{proof}

\usepackage{fancyhdr}
\usepackage{graphicx}
\usepackage{amsmath}

\pagestyle{fancy}
\lfoot{jalrod@ciencias.unam.mx}
\rfoot{jeanpaul@ciencias.unam.mx}

\begin{document}

\begin{flushright}
    Almeida Rodríguez Jerónimo\\
    Ruiz Melo Jean Paul
\end{flushright}

\begin{center}
    {\LARGE Lectura 6}\\
    {\LARGE Big Data}
\end{center}

En el siglo XXI muchos de los avances cinetíficos y descubrimientos que se han
hecho son gracias a una ``nueva'' rama de las matemáticas llamada teoría de
redes, la cual se ocupa de estudiar las relaciones entre los conjuntos de
información (nodos).\\

Según el artículo, una red es una entidad de partes interconectadas en dónde las
conecciones son más importantes que las partes en sí. Esto está definido por una
propiedad llamada complejidad que significa que todo está conectado entre sí, lo
cual puede causar variso problemas. A pesar de que este tipo
de estructuras y su estudio comenzaron a popularizarse con el desarrollo de
internet, las computadoras con grandes capacidades de cómputo, etre otras
tecnologías, estas se pueden encontrar casi en todos los campos de estudio;
desde interacción entre átomos hasta comercio mundial, pasando por interacción
molecular, la cadena alimenticia, redes áereas, etc.\\

Eventualmente, al estudiar estos sistemas uno se da cuenta de que se vuelven
extremadamente complejos; y ahí viene la belleza de la teoría de redes, pues toma
este sistema complejo y lo reduce a elementos y relaciones entre sí. Este poder
de abstracción es el que permitió a personas cómo Henry Beck simplificar el mapa
de red de trenes de Londres estableciendo que la ``redografía'' (netography) es
más importante que la geografía real de la red, o a Kevin Bacon a desarrollar el
juego de ``Los Seis Grados de Kevin Bacon'' que establece que una persona en EUA
está separada de cualquier otra persona por a lo más seis conecciones. Estas
conecciones evolucionaron en la ley universal en redes de ``mundos pequeños''
que permite que el número de saltos que toma ir de un lugar a otro en una red
sea vastamente reducido.\\

Aquí es dónde entra en juego el concepto de ``Big-data'' con el nacimiento de la
internet. El comportamiento de los usuarios en internet permitió a los
científicos descubrir reglas generales de comportamiento cómo los ``Hub's'' que
son nodos por dónde transita mucha información y que controlan el comportamiento
de grandes redes. Analizar estas cantidades
masivas de información ha otorgado a los científicos el poder de observar
sistemas complejos de arriba hacia abajo y encontrar la mejor manera de analizar
información con el fin de utilizarla mejor. El reto ahora es balancear la
dualidad que otorga esta nueva manera de entender el mundo.

\end{document}
