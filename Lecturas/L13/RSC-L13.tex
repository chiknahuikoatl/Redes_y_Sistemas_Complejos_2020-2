<<<<<<< HEAD
\documentclass[a4paper, 11pt]{report}
=======
\documentclass[a4paper, 12pt]{report}
>>>>>>> 81f8f57d2a9647a1cf8f69927d526467cf73039f
\usepackage[spanish]{babel}
\usepackage[utf8]{inputenc}
\usepackage{textcomp}
\usepackage{booktabs}
\usepackage{amssymb}
\usepackage{bussproofs}
\usepackage{proof}

\usepackage{geometry}
\geometry{
 top=4cm,
}

\usepackage{fancyhdr}
\usepackage{graphicx}
\usepackage{amsmath}

\pagestyle{fancy}
\lhead{jalrod@ciencias.unam.mx}
\rhead{jeanpaul@ciencias.unam.mx}

\begin{document}

\begin{flushright}
    Almeida Rodríguez Jerónimo\\
    Ruiz Melo Jean Paul
\end{flushright}

\begin{center}
    {\LARGE Lectura 13}\\
    {\LARGE A Dynamical Model for the Air Transportation Network.}
\end{center}

Aunque la red de transporte aéreo es muy compleja y hay mucha información en
muchas dimensiones bajo las que se puede estudiar, una manera de estudiarla es
con redes comlejas. El problema es que el modelo de redes es estático y la red
de transporte es dinámica, en particular, nos conscierne el tiempo. Para
resolver este problema, nos apoyamos de la familia de redes llamada redes
calendarizadas que permiten conexiones gracias a información externa.\\

Primero se inicia creando un modelo matemático de la red con nodos y aristas
dirigidas. Después se añaden varios nodos secundarios que se utilizan para
simular retrazo de la conexión entre cualesquiera dos nodos y se utilizan
``agentes'' para apoyar la simulación. La matriz de adyacencia de esta nueva red
se divide en 4:
\begin{itemize}
    \item{Persistencia: permite a los agentes en nodos primarios permanecer ahí
        indefinidamente.}
    \item{Activación: representa el calendario. Un agente se mueve de un nodo
        primario a un nodo secundario.}
    \item{Recepción: mueve un agente de un nodo secundario a uno primario.}
    \item{Transferencia: mueve un agente de un nodo secundario a otro. Simula
        el paso del tiempo.}
\end{itemize}
Luego, la matriz de adyacencia estática se mantiene de paso en paso y la matriz
de activación cambia.\\

Para simular el movimiento de una manera más realista, se le da a cada agente un
valor de atracción para moverse a otro nodo. Después, calculamos el costo del
viaje según el costo de tiempo y los pasajeros dispuestos a viajar en dicho
tiempo. Se usa entonces un algoritmo greedy para sumar a la red la conexión
entre cualesquiera dos nodos que sea de menor costo. Finalmente, corremos varios
pasos de la simulación variando los parámetros de costo de vuelo.\\

Podemos observar después de varias iteraciones, y alterando los costos de viaje
de un nodo a otro, las conecciónes y la formación de hubs que se va dando dentro
de la red.

Scheduled networks is a recently developed tool that allows us to
define an algorithm to simulate the growth of a network, in this case
air transportation networks. \\

A network like this has a lot of complex interactions which can make
it difficult to study. However an instrument which helps us with this
is Complex Network theory. \\

These networks use nodes that are connected through hubs following a
given topology. Although powerful, the structure is considered static
in the sense of time. To overcome this problem, we use scheduled
networks which activates nodes based on external information, like
time for example. \\

First, we define a static directed network, in which n nodes are
connected through links in a adjacency matrix A$_{nxn}$, where each
value a$_{ij}$ has a value of one if there exists a link between two
nodes i and j. Now we modify it so that we can include the time
needed to travel to a connection. We add secondary nodes to some of
the primary nodes, which we use as schedulers. These secondary nodes
don't exist in the actual system. The adjacency matrix is now a
differential adjacency matrix. This matrix is constant and holds one
time step, while an activation matrix holds the scheduling
information. \\

The global matrix however, is a matrix of m x m sides, where m is
the sum of primary and secondary nodes. This matrix S is represented
as:\\
$A = [ P \| R ] / [Act \| T]\\$
Where P, R, Act, and T are submatrixes, with P in the top left, R top
right, Act bottom left and T bottom right. We can defome these matrixes as:
\begin{itemize}
    \item P: The original identity matrix
    \item ACT: The schedule matrix, which is updated at each step
    \item R: Moves an agent from a secondary to primary node at the end of a link
    \item T: The transfer matrix which simulates the passage of time
\end{itemize}
We can then chain together these matrix's in order obtain the time
window, which we can represent as:
\begin{center}
    A = A(t) ... A(t + $\delta$ -1) A(t + $\delta$) = \\
    = \prod$^{t+1}_{p=t}$[dA + Act(p)]
\end{center}

We can then simulate the airline transportation as a group of
non-connected nodes, and at each time step we add another
flight to the system. Then to determine the connection we
ask which flight would minimize the global cost using a
greedy algorithm. \\

The results of this produce a hub-and-spoke network in which
all nodes can connect two another nodes in a limited number
of flights, origin to hub, and then hub to destination.
However this system is lost the more nodes there are,
eventually leading into a complex system. \\

However hub-and-spoke networks can break if there is no
defined hub to choose from, or if simply their is too much
noise to allow one to form.

\end{document}
