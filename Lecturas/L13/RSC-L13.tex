\documentclass[a4paper, 11pt]{report}
\usepackage[spanish]{babel}
\usepackage[utf8]{inputenc}
\usepackage{textcomp}
\usepackage{booktabs}
\usepackage{amssymb}
\usepackage{bussproofs}
\usepackage{proof}

\usepackage{geometry}
\geometry{
 top=4cm,
}

\usepackage{fancyhdr}
\usepackage{graphicx}
\usepackage{amsmath}

\pagestyle{fancy}
\lhead{jalrod@ciencias.unam.mx}
\rhead{jeanpaul@ciencias.unam.mx}

\begin{document}

\begin{flushright}
    Almeida Rodríguez Jerónimo\\
    Ruiz Melo Jean Paul
\end{flushright}

\begin{center}
    {\LARGE Lectura 13}\\
    {\LARGE A Dynamical Model for the Air Transportation Network.}
\end{center}

Aunque la red de transporte aéreo es muy compleja y hay mucha información en
muchas dimensiones bajo las que se puede estudiar, una manera de estudiarla es
con redes comlejas. El problema es que el modelo de redes es estático y la red
de transporte es dinámica, en particular, nos conscierne el tiempo. Para
resolver este problema, nos apoyamos de la familia de redes llamada redes
calendarizadas que permiten conexiones gracias a información externa.\\

Primero se inicia creando un modelo matemático de la red con nodos y aristas
dirigidas. Después se añaden varios nodos secundarios que se utilizan para
simular retrazo de la conexión entre cualesquiera dos nodos y se utilizan
``agentes'' para apoyar la simulación. La matriz de adyacencia de esta nueva red
se divide en 4:
\begin{itemize}
    \item{Persistencia: permite a los agentes en nodos primarios permanecer ahí
        indefinidamente.}
    \item{Activación: representa el calendario. Un agente se mueve de un nodo
        primario a un nodo secundario.}
    \item{Recepción: mueve un agente de un nodo secundario a uno primario.}
    \item{Transferencia: mueve un agente de un nodo secundario a otro. Simula
        el paso del tiempo.}
\end{itemize}
Luego, la matriz de adyacencia estática se mantiene de paso en paso y la matriz
de activación cambia.\\

Para simular el movimiento de una manera más realista, se le da a cada agente un
valor de atracción para moverse a otro nodo. Después, calculamos el costo del
viaje según el costo de tiempo y los pasajeros dispuestos a viajar en dicho
tiempo. Se usa entonces un algoritmo greedy para sumar a la red la conexión
entre cualesquiera dos nodos que sea de menor costo. Finalmente, corremos varios
pasos de la simulación variando los parámetros de costo de vuelo.\\

Podemos observar después de varias iteraciones, y alterando los costos de viaje
de un nodo a otro, las conecciónes y la formación de hubs que se va dando dentro
de la red.


\end{document}
