\documentclass[a4paper, 11pt]{report}
\usepackage[spanish]{babel}
\usepackage[utf8]{inputenc}
\usepackage{textcomp}
\usepackage{booktabs}
\usepackage{amssymb}
\usepackage{bussproofs}
\usepackage{proof}

\usepackage{geometry}
\geometry{
 top=4cm,
}

\usepackage{fancyhdr}
\usepackage{graphicx}
\usepackage{amsmath}

\pagestyle{fancy}
\lhead{jalrod@ciencias.unam.mx}
\rhead{jeanpaul@ciencias.unam.mx}

\begin{document}

\begin{flushright}
    Almeida Rodríguez Jerónimo\\
    Ruiz Melo Jean Paul
\end{flushright}

\begin{center}
    {\LARGE Lectura 13}\\
    {\LARGE A Dynamical Model for the Air Transportation Network.}
\end{center}

Aunque la red de transporte aéreo es muy compleja y hay mucha información en
muchas dimensiones bajo las que se puede estudiar, una manera de hacerlo es
con redes complejas. El problema es que el modelo de redes es estático y la red
de transporte es dinámica, en particular, nos importa el tiempo. Para
resolver este problema, nos apoyamos de la familia de redes llamada redes
calendarizadas que muestran el crecimiento de una red, las cuales
activan nodos basándose en información exerna cómo el tiempo.\\

Primero, definimos un modelo matemático de una red dirigida estática. Después se añaden varios nodos secundarios que se utilizan para
simular el paso del tiempo entre cualesquiera dos nodos y se utilizan
``agentes'' para apoyar la simulación. La matriz de adyacencia es diferencial y muestra un paso mientras que la de activación almacena la información de calendarizado. Esta última red es dada por la fórmula $A = \frac{[ P | R ]}{[Act | T]}$ dónde:
\begin{itemize}
    \item{(P)ersistencia: permite a los agentes en nodos prite ls envíomarios permanecer ahí
        indefinidamente.}
    \item{(Act)ivación: representa el calendario. Un agente se mueve de un nodo
        primario a un nodo secundario.}
    \item{(R)ecepción: mueve un agente de un nodo secundario a uno primario.}
    \item{(T)ransferencia: mueve un agente de un nodo secundario a otro. Simula
        el paso del tiempo.}
\end{itemize}

Podemos encadenar matrices para obtener la ventana de tiempo, la cual es representada así:
\begin{center}
    A = A(t) ... A(t + $\delta$ -1) A(t + $\delta$) = \\
    = $\prod^{t+1}_{p=t}$[dA + Act(p)]
\end{center}

Podemos simular el transporte de la aerolínea cómo un grupo de nodos
no conectados. En cada momento añadimos un vuelo al sistema y para simularlo de una manera más realista, se le da a cada agente un
valor de atracción para moverse de un nodo a otro. Después, calculamos con un algoritmo greedy el costo del
viaje según el tiempo y los pasajeros dispuestos a viajar en cada momento. \\

Este resultado produce una red de ``hub-and-spoke'' en dónde todos los nodos se pueden conectar a cualesquiera otros dos nodos en un número limitado de vuelos, del origen al ``hub'' y de ahí a su destino. Este sistema eventualmente se convierte en un sistema complejo entre más nodos halla. No obstante, si no hay un ``hub''
definido para elegir, el ruido es demasiado cómo para formar uno.

\end{document}
