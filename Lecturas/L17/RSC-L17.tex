\documentclass[a4paper, 12pt]{report}
\usepackage[spanish]{babel}
\usepackage[utf8]{inputenc}
\usepackage{textcomp}
\usepackage{booktabs}
\usepackage{amssymb}
\usepackage{bussproofs}
\usepackage{proof}

\usepackage{geometry}
\geometry{
 top=4cm,
}

\usepackage{fancyhdr}
\usepackage{graphicx}
\usepackage{amsmath}

\pagestyle{fancy}
\lhead{jalrod@ciencias.unam.mx}
\rhead{jeanpaul@ciencias.unam.mx}

\begin{document}

\begin{flushright}
    Almeida Rodríguez Jerónimo\\
    Ruiz Melo Jean Paul
\end{flushright}

\begin{center}
    {\LARGE Lectura 16}\\
    {\LARGE Forests Emerge as a Major Overlooked Climate Factor.}
\end{center}
Debido a que el mundo está en constante cambio, por ejemplo los movimentos de las placas tectónicas,
puede suceder que varios mapas no estén actualizados. Por ejemplo, si estas en medio
de la calle, y buscas en Google Maps para ver tu posicion, puede mostrar que estás
a una cierta distancia de tu posición real.\\

En general, el rango de diferencia puede ser de 1 a 50 metros. Esto occure debido a que se
está intentando incrustar imagenes desde el espacio a una cuadricula de longitud y latitud. Para ayudar
con esto se usan marcas de encuesta, las cuales son revisadas para ver si aún estan en
la posición que deben estar y ajustar según corresponda. Agencias como la NGS revisan a estos marcadores con poca frecuencia, debido a que no tienen los recursos para
hacerlo frecuentemente.\\

Pero hay otros errores que pueden resultar en inexactitudes. Uno de estos
es que los datos se tienen que poner sobre un modelo de la tierra. Aquí es dónde la actividad
de las placas
tectónicas se puede notar.\\

Hay dos mapas principales en Estados Unidos (NAD 83 y NAD 84), dónde la mayoria
son basados en NAD 83 y GPS. Google Maps está basado en NAD 84. El NAD 83 es usado con un propósito civil y está enfocado en solo Estados Unidos, mientras que
el NAD 84, a pesar de perder exactitud, da una generalización de toda la tierra.
El cambio a NAD 83 fue muy grande ya que se estaba actualizando con los mapas militares viejos y el actual. Pero
desde ese momento empezaron a separarse de nuevo. \\

Actualmente se quiere tener una sistema que pueda actualizar los mapas en tiempo real,
no cada 3 años como se hace actualmente. Ademas el mapa actual tiene su centro de la
tierra mal por 2 metros. Esto se piensa actualizar durante este año. \\

Aunque para proyectos pequeños no es muy importante, para proyectos grandes si importa que
Norte America se esta moviendo algunas pulgadas cada año. Esto se incrementa con los
terremotos, ya que se puede notar muy bien que hubo un movimiento entre las placas
despues de un terremoto grande o de varios pequeños.
\end{document}
