\documentclass[a4paper, 12pt]{report}
\usepackage[spanish]{babel}
\usepackage[utf8]{inputenc}
\usepackage{textcomp}
\usepackage{booktabs}
\usepackage{amssymb}
\usepackage{bussproofs}
\usepackage{proof}

\usepackage{geometry}
\geometry{
 top=4cm,
}

\usepackage{fancyhdr}
\usepackage{graphicx}
\usepackage{amsmath}

\pagestyle{fancy}
\lhead{jalrod@ciencias.unam.mx}
\rhead{jeanpaul@ciencias.unam.mx}

\begin{document}

\begin{flushright}
    Almeida Rodríguez Jerónimo\\
    Ruiz Melo Jean Paul
\end{flushright}

\begin{center}
    {\LARGE Lectura 18}\\
    {\LARGE Soil’s Microbial Market Shows the Ruthless Side of Forests.}
\end{center}

Uno de los avances más grandes para entender cómo funcionan los ecosistemas es
analizarlos cómo si fueran parte de un mismo organismo. Lynn Margulis y James
Lovelock proponen algo similar en la hipótesis Gaia al proponer a la Tierra cómo
un organismo unificado y auto regulado.\\

Desde hace aproximadamente 30 años se empezaron a estudiar las interconecciones
entre individuos de un ecosistema. Por ejemplo, Suzanne Simmard empezó a
estudiar cómo se transmitía radiación entre árboles. Esto muestra el nivel de
cooperación que existe entre los distintos niveles de organismos. El problema
que surge con este tipo de estudios es que no podemos ver las redes que están
bajo tierra ni a los microorganismos que las componen.\\

Simmard se enfocó en entender la manera de interactuar de una red de hongos
según las condiciones del medio ambiente en el que se encuentran. Al evolucionar
plantas, microbios y hongos ``juntos'' hay un cierto nivel de ``intercambio'' de
recursos que se da entre estos distintos organismos.\\

Simmard decidió optar por un acercamiento tipo economía para entender cómo una
parte
de la red de hongos se beneficia según los nutrientes que tenga disponibles, que
pueda dar a su planta más cercana y que pueda recibir de ella. El problema de
replicar este tipo de interacciones en un laboratorio es que, al ambiente estar
controlado, cosas que en la naturaleza suceden, no lo hacen en las pruebas y que
el intercambio solamente se da con una planta. Aún
así, lo que hizo Simmard fue poner de un lado muchos recursos que los hongos
pueden procesar y la planta se puede beneficiar de ellos y del otro lado pocos,
creando así una zona rica y una zona pobre. Los hongos ``intercambian'' los
nutrientes que pueden producir con los que produce la planta. Simmard notó que
los hongos en la zona rica transferían sus nutrientes hacia la zona pobre,
haciendo que esta última tuviera un mayor desarrollo. Se cree que esto se debe
gracias a que los hongos de la zona pobre tienen una cuota más alta para
entregarle sus recursos
a la planta más cercana obteniendo más recursos de la misma.\\

Este estudio en particular ha mostrado que una comunidad de hongos, a pesar de
ser organismos simples puede involucrarse en comportamientos complejos y la
capacidad que tiene una red de organismos (por más sencillos que aparenten ser)
para comunicarse y transmitir información y/o recursos.

\end{document}
