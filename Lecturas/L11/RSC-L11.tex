\documentclass[a4paper, 12pt]{report}
\usepackage[spanish]{babel}
\usepackage[utf8]{inputenc}
\usepackage{textcomp}
\usepackage{booktabs}
\usepackage{amssymb}
\usepackage{bussproofs}
\usepackage{proof}

\usepackage{geometry}
\geometry{
 top=4cm,
}

\usepackage{fancyhdr}
\usepackage{graphicx}
\usepackage{amsmath}

\pagestyle{fancy}
\lhead{jalrod@ciencias.unam.mx}
\rhead{jeanpaul@ciencias.unam.mx}

\begin{document}

\begin{flushright}
    Almeida Rodríguez Jerónimo\\
    Ruiz Melo Jean Paul
\end{flushright}

\begin{center}
    {\LARGE Lectura 11}\\
    {\LARGE Human Disease Net}
\end{center}
El mayoria de los estudios hechos sobre enfermedades eran con un encoque singular,
pero ahora se quiere hacerlo con un enfoque de alto nivel, y ver los relaciones entre
ellos. Lo que se hace es ver el relacion entre 'fenomo de enfermedad humano' y
'genoma de la enfermedad'. Se construye una grafica bipartita donde un lado
son los trastornos geneticos y el otro lado son todo los enfermedades en el
genomo humano, tales que son conectados si un mutacion entre un gene y enfermedad
existe.  \\

De esto, generamos un HDN, donde los nodos son trastornos y estan conectados
si comparten un gene que una mutacion esta asociado con los dos. Y el DGN,
donde los nodos son genomos de enfermedades y dos nodos son conectados
si estan asociado con el mismo trastorno. \\

Lo que podemos notar con HDN es que la mayoria de los trastornos tienen una conexion
a otro trastorno, y un gran parte de esos forman un componente grande, lo cual
nos dice que los origines geneticos de enfermedades son compartidos con otros.  \\

En DGN, podemos pensar sus conexiones como un relación fenotípica. La mayoria
de los genes estan conectados, mientras que mas de la mitad pertenecen a un
componente grande, pero donde el numero de genes involucrados en multiplos
enfermedades bajo. \\

Para ver que el topolgia de los dos no es random, movemos el associacion entre
los genes y trastornos mientras quedando con el mismos numero de aristas en
el red bipartita. Lo que se nota que el componente gigante de los dos crece mucho,
esto nos surgiere que hay un agrupación fisiopatológica entre los genes y
trastornos. \\

Antes se pensaba que los enfermedades humanos tendran una tendencia a tener
hub encriptado, pero se resulto que no, postulando preguntas como cual es el
role del red cellular de enfermedades, y si son mas probables a encriptar
los hubs. Lo que se encuentro es que los protenias de enfermedad tienen un alta probabilidad
de encriptarse comparado a los proteinas normales que son mas bajos.
\\

Pero el relacion entre genes y los protenias guarda relaciones entre los genese de enfermedades.
Primero dividimos los genese en essenciales y no-essenciales, tal que los proteinas
essenciales tienen la tendencia de ser asociado a hubs, mas que los proteians de enfermedades,
pero cuando se ve la dependencia de las proteians de los enfermedades su relacion con hubs
desaparece completamente, lo cual dice que solo era por los genes esenciales. \\

Despeuse de algunas confirmaciones, se puede notar que los genes de enfermedades no tienen
una posicion importante en los sistemas cellular, mientras que los genes esenciales tienen
un roll importante en el red.

\end{document}
