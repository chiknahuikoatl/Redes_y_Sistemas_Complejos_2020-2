\documentclass[a4paper, 12pt]{report}
\usepackage[spanish]{babel}
\usepackage[utf8]{inputenc}
\usepackage{textcomp}
\usepackage{booktabs}
\usepackage{amssymb}
\usepackage{bussproofs}
\usepackage{proof}

\usepackage{geometry}
\geometry{
 top=4cm,
}

\usepackage{fancyhdr}
\usepackage{graphicx}
\usepackage{amsmath}

\pagestyle{fancy}
\lhead{jalrod@ciencias.unam.mx}
\rhead{jeanpaul@ciencias.unam.mx}

\begin{document}

\begin{flushright}
    Almeida Rodríguez Jerónimo\\
    Ruiz Melo Jean Paul
\end{flushright}

\begin{center}
    {\LARGE Lectura 11}\\
    {\LARGE Human Disease Net}
\end{center}

La mayoría de los estudios hechos sobre enfermedades tenían un enfoque singular,
pero ahora se quiere darle un enfoque de alto nivel y ver las relaciones entre
ellas. Lo que se hace es ver la relación entre ``genoma de enfermedad humano'' y
``genoma de la enfermedad''. Se construye una gráfica bipartita dónde de un lado
están los trastornos geneticos y del otro lado todas las enfermedades en el
genoma humano, tales que si una mutación entre un gen y la enfermedad existe,
entonces se conectan.  \\

Así, generamos una HDN, donde los nodos son trastornos y están conectados
si comparten un gen que tenga una misma mutacion y la DGN,
donde los nodos son genomas de enfermedades y dos nodos son conectados
si estan asociado con el mismo trastorno. \\

Lo que podemos notar con la HDN es que la mayoria de los trastornos tienen una conexión
a otro trastorno y una gran parte de esos forman una componente grande, lo cual
nos dice que los orígines genéticos de algunas enfermedades son compartidas con
otras enfermedades.  \\

Para ver que la topolgía de ambos no es aleatoria, movemos la asociación entre
los genes y trastornos mientras nos quedamos con el mismo número de aristas en
la red bipartita. Lo que se nota es que la componente gigante de los dos crece mucho,
esto nos surgiere que hay un agrupación fisiopatológica entre genes y
trastornos. \\

Antes se pensaba que las enfermedades humanas tendrían una tendencia a tener un
hub encriptado postulando preguntas como cual es el
rol de la red celular de enfermedades, y si son mas probables de encriptar
los hubs, pero resulto que no. Lo que se encontró es que las proteínas de
enfermedades tienen una alta
probabilidad de encriptarse comparada con las proteinas normales.\\

Pero la relacion entre genes y protenias se mantiene entre los genes de enfermedades.
Primero dividimos los genes en esenciales y no-esenciales, tal que las proteinas
esenciales tienen mayor tendencia de ser asociadas a hubs que las proteinas de enfermedades,
pero cuando se ve la dependencia de las proteinas de las enfermedades su relacion con hubs
desaparece completamente, lo cual dice que solo era por los genes esenciales. \\

\end{document}
