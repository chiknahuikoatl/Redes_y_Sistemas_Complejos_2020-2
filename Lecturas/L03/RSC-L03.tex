\documentclass[a4paper, 12pt]{report}
\usepackage[spanish]{babel}
\usepackage[utf8]{inputenc}
\usepackage{textcomp}
\usepackage{booktabs}
\usepackage{amssymb}
\usepackage{bussproofs}
\usepackage{proof}

\usepackage{fancyhdr}
\usepackage{graphicx}
\usepackage{amsmath}

\pagestyle{fancy}
\lfoot{jalrod@ciencias.unam.mx}
\rfoot{jeanpaul@ciencias.unam.mx}

\begin{document}

\begin{flushright}
    Almeida Rodríguez Jerónimo\\
    Ruiz Melo Jean Paul
\end{flushright}

\begin{center}
    {\LARGE Lectura 3}\\
    {\LARGE Redes y Física Estadística.}
\end{center}

Al intnentar modelar sistemas complejos ``vivos'' con redes surge el problema de
que estos sistemas están en cambio constante. La forma que se ha encontrado de
modelarlos a pesar de esta complicación es por medio de la física estadística.\\

El primer artículo habla sobre el tráfico en una ciudad (Madrid). Los nodos son
las intersecciones y el tráfico se simula mediante peso en los arcos de la red.
Esto hace que la red sea una red de flujo. El problema que se presenta entonces
es que el tráfico no es constante todo el tiempo, sino que hay varianza en el
flujo. La manera que se encontró de estimar el tráfico fue mediante el método
del ``Modelo de Congestión Microscópica'' que describe la ecuación:
$$ \Delta q_i = g_i+\sigma_i-d_i$$
Dónde el incremento de vehículos por unidad de tiempo es la suma de la media de
vehículos que llegan, se quedan y salen de la intersección. Esto permite atacar
un problema global (el tráfico en la red de tránsito) de manera local
(estudiando cada intersección).

Por otro lado, podemos aplicar este principio a la propagación de epidemias.
Sabemos que los patrones de propagación de una epidemia dependen de varios
factores cómo la infectividad, la centralidad del primer grado infectado y el
contexto (momento, ambiente, medio, etc) en el que se desarrolla. Después de
esto podemos estudiar el aspecto de la propagación en sí, es decir, qué tan
suceptible es un nodo a ser infectado, a recuperarse de la infección y si se
puede o no volverse a infectar  (cómo es el caso de las enfermedades
estacionarias). Estos últimos factores se calculan según la probabilidad de que
se de cada uno de los escenarios antes mencionados. Según el artículo estas
probabilidades se dan por las siguientes ecuaciones diferenciales del tiempo:
$$ \frac{dl}{dt}=\beta SI-\mu I \ \ \& \ \ \frac{dS}{dt}=\mu I-\beta IS$$
Estas ecuaciones simulan las transiciones de fase de una epidemia dada una
probabilidad de infección.\\

De esta manera podemos simular con cierto grado de certeza ciertos problemas
de redes de sistemas complejos de manera local (dada la probabilidad en un nodo)
y estudiar toda la red.

\end{document}
