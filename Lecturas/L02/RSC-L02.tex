\documentclass[a4paper, 12pt]{report}
\usepackage[spanish]{babel}
\usepackage[utf8]{inputenc}
\usepackage{textcomp}
\usepackage{booktabs}
\usepackage{amssymb}
\usepackage{bussproofs}
\usepackage{proof}

\usepackage{fancyhdr}
\usepackage{graphicx}
\usepackage{amsmath}

\pagestyle{fancy}
\lfoot{jalrod@ciencias.unam.mx}
\rfoot{jeanpaul@ciencias.unam.mx}

\begin{document}

\begin{flushright}
    Almeida Rodríguez Jerónimo\\
    Ruiz Melo Jean Paul
\end{flushright}

\begin{center}
    {\LARGE Lectura 2}\\
    {\LARGE How Network Math Can Help You Make Friends.}
\end{center}

Un nodo es un elemento en una red que representa de manera atómica partes del
objeto de estudio en cuestión. Cómo por ejemplo:
\begin{itemize}
    \item Villa-regular, dónde cada nodo tiene grado a lo más 4.
    \item Villa-aleatoria, dónde la probabilidad de que un nodo se conecte con
        otro es de $\frac{\#\text{nodos}}{\#\text{posibles conecciones}}$, en este
        ejemplo es de $\frac{1}{5}$
\end{itemize}

Estas redes son fáciles de estudiar pero son poco realistas.\\

Cuando un red empieza a ser muy grande, puede ser difícil entender la
información que representa. Cómo el grado de un nodo sólo da información local
debemos usar el grado de todos los nodos para poder tener un idea global del
comportamiento de la red. En la práctica las redes presentan algo llamado
conección preferencial dónde un nodo que tiene muchos vecinos tiene  una mayor
probabilidad de conectarse a otros vecinos. Estos nodos de alto grado son
importantes ya que otorgan una conexión estable en casos que se empieza a romper
la red.\\

Rara vez las distribuciones de los grados son polinomiales; en general son
exponenciales o logarítmicas normales, es decir que en la realidad muy pocos
nodos son ``extremadamente centrales'' con grados muy grandes. Esto importa en
el momento en el que se quieren aplicar teorías a la información que representa la red.


\end{document}
