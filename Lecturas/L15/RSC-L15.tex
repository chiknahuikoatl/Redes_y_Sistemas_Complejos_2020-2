\documentclass[a4paper, 12pt]{report}
\usepackage[spanish]{babel}
\usepackage[utf8]{inputenc}
\usepackage{textcomp}
\usepackage{booktabs}
\usepackage{amssymb}
\usepackage{bussproofs}
\usepackage{proof}

\usepackage{geometry}
\geometry{
 top=4cm,
}

\usepackage{fancyhdr}
\usepackage{graphicx}
\usepackage{amsmath}

\pagestyle{fancy}
\lhead{jalrod@ciencias.unam.mx}
\rhead{jeanpaul@ciencias.unam.mx}

\begin{document}

\begin{flushright}
    Almeida Rodríguez Jerónimo\\
    Ruiz Melo Jean Paul
\end{flushright}

\begin{center}
    {\LARGE Lectura 15}\\
    {\LARGE Defining a historic football team.}
\end{center}

Del dominio de las redes sociales se derivan las redes de pases en el fútbol.
Estas redes dirigidas consisten en medir la cantidad de pases que hay entre
cualesquiera jugadores $i$ y $j$ en un momento dado y analizar la información
que podemos extraer de esta red. El objetivo es encontrar clústers, centralidad
de ciertos jugadores, etc.\\

El caso de estudio es el equipo FCB durante la jornada 2009/2010. Además de
analizar la red de pases, se toma en cuenta la posición del emisor/receptor (de
dónde podemos extraer longitud del pase y distancia de la meta) y el tiempo, el
cual nos dice qué tan rápido se forma una red específica, en este caso se
analiza principalmente la red de 50 pases.\\

Lo que se pudo observar es que el FCB tiene el coeficiente de clustering más
alto, que la centralidad de los jugadores es más o menos la misma, forman la
red de 50 pases más rápido (en promedio 9 mins) que cualquier otro equipo,
tienen el juego más horizontal, es decir, que mantienen los pases de lado a lado
de la cancha más que moverse hacia la otra portería, entre otras propiedades
observables. El FCB resalta bajo estos parámetros con respecto a otros equipos,
lo cual puede ser lo que le de su ventaja estratégica.\\

Aunque algunas de la propiedades anteriores son específicas del FCB (en otros
equipos no otorgan los mismos resultados, en el caso del Valencia FC, a
diferencia del FCB, un coeficiente de dispersión alto resultó en que anotaran
más goles)
se pueden observar propiedades generales, por ejemplo, que cuándo se va a anotar
un gol, el juego es frontal (se carga hacia la portería del equipo que va a ser
anotado) o que el número de pases está directamente relacionado con el tiempo de
posesión del balón, entre otros.\\

Finalmente, se observó que:
\begin{itemize}
\item{el FCB (gracias a su estrategia de juego) tiene la capacidad de limitar el
    espacio jugable de la cancha.}
\item{Hay dos etapas de juego: jugar el balón y recuperarlo. Al:
    \begin{itemize}
    \item{Recuperarlo: Ir directo por el balón y ponerlo en juego.}
    \item{Jugarlo: El objetivo es encontrar el mejor momento para meter gol con
        el apoyo de un clúster de 3 jugadores cercanos y otros dos de apoyo un
        poco más lejos.
    }
    \end{itemize}
}
\end{itemize}

\end{document}
