\documentclass[a4paper, 12pt]{report}
\usepackage[spanish]{babel}
\usepackage[utf8]{inputenc}
\usepackage{textcomp}
\usepackage{booktabs}
\usepackage{amssymb}
\usepackage{bussproofs}
\usepackage{proof}

\usepackage{geometry}
\geometry{
 top=4cm,
}

\usepackage{fancyhdr}
\usepackage{graphicx}
\usepackage{amsmath}

\pagestyle{fancy}
\lhead{jalrod@ciencias.unam.mx}
\rhead{jeanpaul@ciencias.unam.mx}

\begin{document}

\begin{flushright}
    Almeida Rodríguez Jerónimo\\
    Ruiz Melo Jean Paul
\end{flushright}

\begin{center}
    {\LARGE Proyecto}\\
\end{center}
Nuestro proyecto va a consistir en un sistema que simula un juego de rol.
Cómo un juego de rol es complicado, vamos a tomar solo las siguientes
características:
\begin{itemize}
    \item Existen ciudades a las que pertenecen un número
    arbitrario personajes.
    \item Los personajes pueden ser de una de 3 clases, que
    tienen un rol importante al inicio, mitad y final del
    juego.
    \item Existen monstruos que se mueven por el mundo.
    \item Los personajes salen de ciudades para recoger tesoros.
    \item Los monstruos pueden atacar a los personajes.
    \item Si un personaje gana contra un monstro, gana
    experiencia.
    \item Si un personaje gana suficiente experiencia, sube un nivel.
    \item A lo mas 6 personajes pueden unirse para formar un equipo.
    \item Los personajes buscan tesoros para traerlos a las
    ciudades a las que pertenecen. Las ciudades crecen entre más
    tesororos les traigan los personajes.
    \item Si los personajes recogen suficientes tesoros, pueden
    comprar armas para simular el incremento de nivel.
    \item Los monstruos ganan poder através del tiempo.
    \item Los monstruos pueden destruir una ciudad si el
    poder que tienen sobrepasa el tamaño de una ciudad
\end{itemize}
De esto, queremos ver que suele ser mas importante através del tiempo.
El número de personajes (y sus niveles) que pertenecen a una ciudad, los
tipos de clases que consituyen equipos, o la cantidad de tesoros que
se necesitan para que una ciudad pueda crecer y protegerse. \\

Entonces, los datos que quermos tomar son las clases que
consituyen los equipos, sus niveles y la cantidad de tesoros
que se recogen atravez del tiempo. \\

Estamos esperando que las ciudades con un gran numero de
personajes y tesoros sean las que se queden hasta el final
del tiempo, pero tambien queremos saber cómo se afecta el crecimiente de
las ciudades que no tienen muchos tesoros cercanos; si les importa tener muchos
o pocos personajes, etc.

\end{document}
