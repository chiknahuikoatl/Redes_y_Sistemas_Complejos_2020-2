\documentclass[a4paper, 12pt]{report}
\usepackage[spanish]{babel}
\usepackage[utf8]{inputenc}
\usepackage{textcomp}
\usepackage{booktabs}
\usepackage{amssymb}
\usepackage{bussproofs}
\usepackage{proof}

\usepackage{geometry}
\geometry{
 top=4cm,
}

\usepackage{fancyhdr}
\usepackage{graphicx}
\usepackage{amsmath}

\pagestyle{fancy}
\lhead{jalrod@ciencias.unam.mx}
\rhead{jeanpaul@ciencias.unam.mx}

\begin{document}

\begin{flushright}
    Almeida Rodríguez Jerónimo\\
    Ruiz Melo Jean Paul
\end{flushright}

\begin{center}
    {\LARGE Proyecto}\\
\end{center}
Nuestor proyecto va a consistir en un sistema que simula un juego de role.
Como un juego de role es complicado, vamos a solo tomar los siguientes
caracteristicas. 
\begin{itemize}
    \item Existen ciudades a que pertenecen numero 
    arbitrarios caracters.
    \item Los caracteres pueden ser de uno de 3 clases, que
    tienen un role importante al inicio, medio y final del
    tiempo
    \item Existen monstros que mueven por el mundo
    \item Los caracters salen de ciudades para recoger tesoros
    \item Los monstros pueden attacar a los caracters
    \item Si un caracter gana contra un monstro, gana
    experiencia
    \item Si un caracter gana suficiente experiencia, gana un nivel
    \item A lo mas 6 caracters pueden unirse para formar un equipo
    \item Los caracteres buscan tesoros para traer los a los
    ciudades que pertenecen, dode los ciudades crecen con mas
    tesororos que les dan los caracters
    \item Si los caracters recogen suficiente tesoros, pueden
    comprar armas para simular la ganacia de un nivel
    \item Los monstros ganan poder atravez del tiempo
    \item Los monstros pueden destruir un ciudad si el
    poder que tienen sobre pasa que tan grande es una ciudad
\end{itemize}
De esto, queremos ver que suele ser mas importante atravez del tiempo.
Los niveles y numero de los caracteres que pertenecen a un ciudad, los
tipos de clases que consituyen equipos, o la cantidad de tesoros que
se entrega para que puede crecer y protegerse una ciudad. \\

Entonces los datos que quermos tomar son los clases que
consituyen los equipos, sus niveles y la cantidad de tesoros
que se recogen atravez del tiempo. \\

Estamos esperando que los ciudades con un gran numero de
caracters y tesoros serian los que se quedan hasta el final
del tiempo, pero tambien quermos saber si los ciudades
que no tienen muchos tesoros cercanos les importa tener mas
caracters o pocos para que pueden tener mas experiencia.

\end{document}
